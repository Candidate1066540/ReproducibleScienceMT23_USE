% Options for packages loaded elsewhere
\PassOptionsToPackage{unicode}{hyperref}
\PassOptionsToPackage{hyphens}{url}
%
\documentclass[
]{article}
\usepackage{amsmath,amssymb}
\usepackage{iftex}
\ifPDFTeX
  \usepackage[T1]{fontenc}
  \usepackage[utf8]{inputenc}
  \usepackage{textcomp} % provide euro and other symbols
\else % if luatex or xetex
  \usepackage{unicode-math} % this also loads fontspec
  \defaultfontfeatures{Scale=MatchLowercase}
  \defaultfontfeatures[\rmfamily]{Ligatures=TeX,Scale=1}
\fi
\usepackage{lmodern}
\ifPDFTeX\else
  % xetex/luatex font selection
\fi
% Use upquote if available, for straight quotes in verbatim environments
\IfFileExists{upquote.sty}{\usepackage{upquote}}{}
\IfFileExists{microtype.sty}{% use microtype if available
  \usepackage[]{microtype}
  \UseMicrotypeSet[protrusion]{basicmath} % disable protrusion for tt fonts
}{}
\makeatletter
\@ifundefined{KOMAClassName}{% if non-KOMA class
  \IfFileExists{parskip.sty}{%
    \usepackage{parskip}
  }{% else
    \setlength{\parindent}{0pt}
    \setlength{\parskip}{6pt plus 2pt minus 1pt}}
}{% if KOMA class
  \KOMAoptions{parskip=half}}
\makeatother
\usepackage{xcolor}
\usepackage[margin=1in]{geometry}
\usepackage{color}
\usepackage{fancyvrb}
\newcommand{\VerbBar}{|}
\newcommand{\VERB}{\Verb[commandchars=\\\{\}]}
\DefineVerbatimEnvironment{Highlighting}{Verbatim}{commandchars=\\\{\}}
% Add ',fontsize=\small' for more characters per line
\usepackage{framed}
\definecolor{shadecolor}{RGB}{248,248,248}
\newenvironment{Shaded}{\begin{snugshade}}{\end{snugshade}}
\newcommand{\AlertTok}[1]{\textcolor[rgb]{0.94,0.16,0.16}{#1}}
\newcommand{\AnnotationTok}[1]{\textcolor[rgb]{0.56,0.35,0.01}{\textbf{\textit{#1}}}}
\newcommand{\AttributeTok}[1]{\textcolor[rgb]{0.13,0.29,0.53}{#1}}
\newcommand{\BaseNTok}[1]{\textcolor[rgb]{0.00,0.00,0.81}{#1}}
\newcommand{\BuiltInTok}[1]{#1}
\newcommand{\CharTok}[1]{\textcolor[rgb]{0.31,0.60,0.02}{#1}}
\newcommand{\CommentTok}[1]{\textcolor[rgb]{0.56,0.35,0.01}{\textit{#1}}}
\newcommand{\CommentVarTok}[1]{\textcolor[rgb]{0.56,0.35,0.01}{\textbf{\textit{#1}}}}
\newcommand{\ConstantTok}[1]{\textcolor[rgb]{0.56,0.35,0.01}{#1}}
\newcommand{\ControlFlowTok}[1]{\textcolor[rgb]{0.13,0.29,0.53}{\textbf{#1}}}
\newcommand{\DataTypeTok}[1]{\textcolor[rgb]{0.13,0.29,0.53}{#1}}
\newcommand{\DecValTok}[1]{\textcolor[rgb]{0.00,0.00,0.81}{#1}}
\newcommand{\DocumentationTok}[1]{\textcolor[rgb]{0.56,0.35,0.01}{\textbf{\textit{#1}}}}
\newcommand{\ErrorTok}[1]{\textcolor[rgb]{0.64,0.00,0.00}{\textbf{#1}}}
\newcommand{\ExtensionTok}[1]{#1}
\newcommand{\FloatTok}[1]{\textcolor[rgb]{0.00,0.00,0.81}{#1}}
\newcommand{\FunctionTok}[1]{\textcolor[rgb]{0.13,0.29,0.53}{\textbf{#1}}}
\newcommand{\ImportTok}[1]{#1}
\newcommand{\InformationTok}[1]{\textcolor[rgb]{0.56,0.35,0.01}{\textbf{\textit{#1}}}}
\newcommand{\KeywordTok}[1]{\textcolor[rgb]{0.13,0.29,0.53}{\textbf{#1}}}
\newcommand{\NormalTok}[1]{#1}
\newcommand{\OperatorTok}[1]{\textcolor[rgb]{0.81,0.36,0.00}{\textbf{#1}}}
\newcommand{\OtherTok}[1]{\textcolor[rgb]{0.56,0.35,0.01}{#1}}
\newcommand{\PreprocessorTok}[1]{\textcolor[rgb]{0.56,0.35,0.01}{\textit{#1}}}
\newcommand{\RegionMarkerTok}[1]{#1}
\newcommand{\SpecialCharTok}[1]{\textcolor[rgb]{0.81,0.36,0.00}{\textbf{#1}}}
\newcommand{\SpecialStringTok}[1]{\textcolor[rgb]{0.31,0.60,0.02}{#1}}
\newcommand{\StringTok}[1]{\textcolor[rgb]{0.31,0.60,0.02}{#1}}
\newcommand{\VariableTok}[1]{\textcolor[rgb]{0.00,0.00,0.00}{#1}}
\newcommand{\VerbatimStringTok}[1]{\textcolor[rgb]{0.31,0.60,0.02}{#1}}
\newcommand{\WarningTok}[1]{\textcolor[rgb]{0.56,0.35,0.01}{\textbf{\textit{#1}}}}
\usepackage{graphicx}
\makeatletter
\def\maxwidth{\ifdim\Gin@nat@width>\linewidth\linewidth\else\Gin@nat@width\fi}
\def\maxheight{\ifdim\Gin@nat@height>\textheight\textheight\else\Gin@nat@height\fi}
\makeatother
% Scale images if necessary, so that they will not overflow the page
% margins by default, and it is still possible to overwrite the defaults
% using explicit options in \includegraphics[width, height, ...]{}
\setkeys{Gin}{width=\maxwidth,height=\maxheight,keepaspectratio}
% Set default figure placement to htbp
\makeatletter
\def\fps@figure{htbp}
\makeatother
\setlength{\emergencystretch}{3em} % prevent overfull lines
\providecommand{\tightlist}{%
  \setlength{\itemsep}{0pt}\setlength{\parskip}{0pt}}
\setcounter{secnumdepth}{-\maxdimen} % remove section numbering
\ifLuaTeX
  \usepackage{selnolig}  % disable illegal ligatures
\fi
\IfFileExists{bookmark.sty}{\usepackage{bookmark}}{\usepackage{hyperref}}
\IfFileExists{xurl.sty}{\usepackage{xurl}}{} % add URL line breaks if available
\urlstyle{same}
\hypersetup{
  pdftitle={ReproducibleScienceMT23HomeWorkDec2023},
  pdfauthor={Candidate 1066540},
  hidelinks,
  pdfcreator={LaTeX via pandoc}}

\title{ReproducibleScienceMT23HomeWorkDec2023}
\author{Candidate 1066540}
\date{2023-12-06}

\begin{document}
\maketitle

\hypertarget{question-01-data-visualisation-for-science-communication}{%
\section{\# QUESTION 01: Data Visualisation for Science
Communication}\label{question-01-data-visualisation-for-science-communication}}

\hypertarget{a-provide-your-figure-here}{%
\subsubsection{a) Provide your figure
here:}\label{a-provide-your-figure-here}}

\begin{Shaded}
\begin{Highlighting}[]
\CommentTok{\# There is no need to provide the code for your bad figure, just use echo=FALSE so the code is hidden. Make sure your figure is visible after you knit it.}

\CommentTok{\#install.packages("tinytex")}
\CommentTok{\#library(tinytex) \# to allow kniting to a PDF.}

\CommentTok{\# packages installed via interface.}
\CommentTok{\# load the packages in via library() function:}
\FunctionTok{library}\NormalTok{(ggplot2)}
\FunctionTok{library}\NormalTok{(palmerpenguins)}
\end{Highlighting}
\end{Shaded}

\begin{verbatim}
## Warning: package 'palmerpenguins' was built under R version 4.3.2
\end{verbatim}

\begin{Shaded}
\begin{Highlighting}[]
\FunctionTok{library}\NormalTok{(janitor)}
\end{Highlighting}
\end{Shaded}

\begin{verbatim}
## Warning: package 'janitor' was built under R version 4.3.2
\end{verbatim}

\begin{verbatim}
## 
## Attaching package: 'janitor'
\end{verbatim}

\begin{verbatim}
## The following objects are masked from 'package:stats':
## 
##     chisq.test, fisher.test
\end{verbatim}

\begin{Shaded}
\begin{Highlighting}[]
\FunctionTok{library}\NormalTok{(dplyr)}
\end{Highlighting}
\end{Shaded}

\begin{verbatim}
## Warning: package 'dplyr' was built under R version 4.3.2
\end{verbatim}

\begin{verbatim}
## 
## Attaching package: 'dplyr'
\end{verbatim}

\begin{verbatim}
## The following objects are masked from 'package:stats':
## 
##     filter, lag
\end{verbatim}

\begin{verbatim}
## The following objects are masked from 'package:base':
## 
##     intersect, setdiff, setequal, union
\end{verbatim}

\begin{Shaded}
\begin{Highlighting}[]
\CommentTok{\# check the data is there:}
\FunctionTok{head}\NormalTok{(penguins)}
\end{Highlighting}
\end{Shaded}

\begin{verbatim}
## # A tibble: 6 x 8
##   species island    bill_length_mm bill_depth_mm flipper_length_mm body_mass_g
##   <fct>   <fct>              <dbl>         <dbl>             <int>       <int>
## 1 Adelie  Torgersen           39.1          18.7               181        3750
## 2 Adelie  Torgersen           39.5          17.4               186        3800
## 3 Adelie  Torgersen           40.3          18                 195        3250
## 4 Adelie  Torgersen           NA            NA                  NA          NA
## 5 Adelie  Torgersen           36.7          19.3               193        3450
## 6 Adelie  Torgersen           39.3          20.6               190        3650
## # i 2 more variables: sex <fct>, year <int>
\end{verbatim}

\begin{Shaded}
\begin{Highlighting}[]
\CommentTok{\# Now, make a correct but misleading plot.}
\FunctionTok{ggplot}\NormalTok{(}\AttributeTok{data=}\NormalTok{penguins, }\FunctionTok{aes}\NormalTok{(}\AttributeTok{x=}\NormalTok{ flipper\_length\_mm, }\AttributeTok{y =}\NormalTok{ body\_mass\_g)) }\SpecialCharTok{+}
  \FunctionTok{geom\_point}\NormalTok{() }
\end{Highlighting}
\end{Shaded}

\begin{verbatim}
## Warning: Removed 2 rows containing missing values (`geom_point()`).
\end{verbatim}

\includegraphics{ReproducibleScienceMT23HomeworkDec2023_files/figure-latex/bad figure code-1.pdf}
\# \# \# b)

The plot produced here is misleading, as while the data is shown for all
three penguin species in the palmerpenguins package - Adelie, Gentoo and
Chinstrap, the data points for one species cannot be identified from
those of either of the other two species present on the plot.This is
grossly misleading for the reader and a demonstration of poor data
communication.To better communicate the data visually, the data points
could be colour coded for each penguin species, thus enabling patterns
within and between species to be visually discerned by the reader.
Colour coding in plots can be a very useful tool (though care should be
taken for colour blind people when choosing colours for plots). An
effective use of colour is that shown by the bar charts communicating
data on attempts to reproduce a work in the article by M.Baker.

The above scatter plot here also lacks mention in itself that there are
three species of penguin being represented by and a title, again
misleading the reader.

Reference:

Baker,M., (2016). Is there a reproducibility crisis? A Nature survey
lifts the lid on how researchers view the `crisis' rocking science and
what they think will help , Nature , Vol.535, pp.452-454.

\hypertarget{question-2-data-pipeline}{%
\section{\# Question 2: Data Pipeline}\label{question-2-data-pipeline}}

\hypertarget{introduction}{%
\subsubsection{Introduction}\label{introduction}}

Penguins are small to large flightless birds mostly found in the
southern hemisphere (1). Data has been collected on a range of variables
for three penguin species: Adelie, Gentoo, and Chinstrap (the
palmerpenguins dataset).

\begin{Shaded}
\begin{Highlighting}[]
\CommentTok{\# Make sure your code prints. }

\CommentTok{\# VITAL: if unsure, be sure  the Rmd being accessed is " ReproducibleScienceMT23HomeworkDec2023.Rmd". }

\FunctionTok{install.packages}\NormalTok{(}\FunctionTok{c}\NormalTok{(}\StringTok{"ggplot2"}\NormalTok{, }\StringTok{"palmerpenguins"}\NormalTok{, }\StringTok{"janitor"}\NormalTok{, }\StringTok{"dplyr"}\NormalTok{)) }\CommentTok{\# Installs all packages in one line. }
\end{Highlighting}
\end{Shaded}

\begin{verbatim}
## Warning: packages 'ggplot2', 'palmerpenguins', 'janitor', 'dplyr' are in use
## and will not be installed
\end{verbatim}

\begin{Shaded}
\begin{Highlighting}[]
\FunctionTok{library}\NormalTok{(ggplot2)}
\FunctionTok{library}\NormalTok{(palmerpenguins)}
\FunctionTok{library}\NormalTok{(janitor)}
\FunctionTok{library}\NormalTok{(dplyr) }\CommentTok{\# loads the required packages for working with the penguin data.}

\FunctionTok{head}\NormalTok{(penguins\_raw) }\CommentTok{\# enables visual inspection of the RAW data. Will need to clean and process later on for better use.}
\end{Highlighting}
\end{Shaded}

\begin{verbatim}
## # A tibble: 6 x 17
##   studyName `Sample Number` Species          Region Island Stage `Individual ID`
##   <chr>               <dbl> <chr>            <chr>  <chr>  <chr> <chr>          
## 1 PAL0708                 1 Adelie Penguin ~ Anvers Torge~ Adul~ N1A1           
## 2 PAL0708                 2 Adelie Penguin ~ Anvers Torge~ Adul~ N1A2           
## 3 PAL0708                 3 Adelie Penguin ~ Anvers Torge~ Adul~ N2A1           
## 4 PAL0708                 4 Adelie Penguin ~ Anvers Torge~ Adul~ N2A2           
## 5 PAL0708                 5 Adelie Penguin ~ Anvers Torge~ Adul~ N3A1           
## 6 PAL0708                 6 Adelie Penguin ~ Anvers Torge~ Adul~ N3A2           
## # i 10 more variables: `Clutch Completion` <chr>, `Date Egg` <date>,
## #   `Culmen Length (mm)` <dbl>, `Culmen Depth (mm)` <dbl>,
## #   `Flipper Length (mm)` <dbl>, `Body Mass (g)` <dbl>, Sex <chr>,
## #   `Delta 15 N (o/oo)` <dbl>, `Delta 13 C (o/oo)` <dbl>, Comments <chr>
\end{verbatim}

\begin{Shaded}
\begin{Highlighting}[]
\CommentTok{\#knitr::opts\_chunk$set(echo = TRUE)}
\end{Highlighting}
\end{Shaded}

\begin{Shaded}
\begin{Highlighting}[]
\CommentTok{\# clean the data. Ensure it is saved to the data folder in the R project, which is called AssignmentDataReproSci in this instance.}
\NormalTok{penguins\_raw}
\end{Highlighting}
\end{Shaded}

\begin{verbatim}
## # A tibble: 344 x 17
##    studyName `Sample Number` Species         Region Island Stage `Individual ID`
##    <chr>               <dbl> <chr>           <chr>  <chr>  <chr> <chr>          
##  1 PAL0708                 1 Adelie Penguin~ Anvers Torge~ Adul~ N1A1           
##  2 PAL0708                 2 Adelie Penguin~ Anvers Torge~ Adul~ N1A2           
##  3 PAL0708                 3 Adelie Penguin~ Anvers Torge~ Adul~ N2A1           
##  4 PAL0708                 4 Adelie Penguin~ Anvers Torge~ Adul~ N2A2           
##  5 PAL0708                 5 Adelie Penguin~ Anvers Torge~ Adul~ N3A1           
##  6 PAL0708                 6 Adelie Penguin~ Anvers Torge~ Adul~ N3A2           
##  7 PAL0708                 7 Adelie Penguin~ Anvers Torge~ Adul~ N4A1           
##  8 PAL0708                 8 Adelie Penguin~ Anvers Torge~ Adul~ N4A2           
##  9 PAL0708                 9 Adelie Penguin~ Anvers Torge~ Adul~ N5A1           
## 10 PAL0708                10 Adelie Penguin~ Anvers Torge~ Adul~ N5A2           
## # i 334 more rows
## # i 10 more variables: `Clutch Completion` <chr>, `Date Egg` <date>,
## #   `Culmen Length (mm)` <dbl>, `Culmen Depth (mm)` <dbl>,
## #   `Flipper Length (mm)` <dbl>, `Body Mass (g)` <dbl>, Sex <chr>,
## #   `Delta 15 N (o/oo)` <dbl>, `Delta 13 C (o/oo)` <dbl>, Comments <chr>
\end{verbatim}

\begin{Shaded}
\begin{Highlighting}[]
\FunctionTok{write.csv}\NormalTok{(penguins\_raw, }\StringTok{"AssignmentDataReproSci/penguins\_raw.csv"}\NormalTok{) }\CommentTok{\# }\AlertTok{NOTE}\CommentTok{: need to ensure "FolderName/TheRawData" is correct. Don\textquotesingle{}t put "data/rawdata" if the folder name is NOT data. Applies to name of data file too.}

\CommentTok{\#knitr::opts\_chunk$set(echo = TRUE)}
\end{Highlighting}
\end{Shaded}

\hypertarget{hypothesis}{%
\subsubsection{Hypothesis}\label{hypothesis}}

Null hypothesis: Culmen depth does not increase with increasing body
mass in the sampled penguin population. Alternate: culmen depth does
change with increasing body mass in the penguin population that was
sampled.

The culmen is an aspect of a bird's beak that can be measured (2). The
work that follows in this report assumes culmen depth is measured where
the beak connects with the rest of the bird's head.

Thought plausible, as greater body mass could imply greater muscle size
and greater muscle attachment may be facilitated by a larger jaw and
culmen , allowing for stronger bites and pecks. By looking at it from
the scale of the whole population that was sampled, some inference cold
be made about selection pressure for stringer bite and peck ability in
these penguins.

It would be of interest to see whether such a relationship exists in
these penguins and whether it holds for other closely related and
non-closely related bird groups.

\hypertarget{statistical-methods}{%
\subsubsection{Statistical Methods}\label{statistical-methods}}

A linear regression was applied, as both variables - body mass and
culmen depth - are numerical. The analysis presented here assumes
conforming to normality and the assumptions of normality.

Correlation between the variables - body mass and culmen depth - is
visually inferred (categorically) from the slope of the regression line.

The hypothesis stated in the introduction section can be formalised with
respect to testing via linear regression (4): - Null: the slope of the
population is equal to 0 - Alternate: the slope of the population does
not equal 0.

\begin{Shaded}
\begin{Highlighting}[]
\CommentTok{\# Make sure your code prints. }

\CommentTok{\# Make sure that your code is made to address the hypotheses being tested. Is vital for so much good science.}

\CommentTok{\# Refer to the bad exploratory plot and code for basics if required.}

\CommentTok{\# Clean the data, now it is saved in the folder AssignmentDataReproSci. }
\NormalTok{penguins\_raw }\OtherTok{\textless{}{-}} \FunctionTok{read.csv}\NormalTok{(}\StringTok{"AssignmentDataReproSci/penguins\_raw.csv"}\NormalTok{) }\CommentTok{\# ENSURE all typed correctly.}

\FunctionTok{names}\NormalTok{(penguins\_raw)}\CommentTok{\# to check the column names.}
\end{Highlighting}
\end{Shaded}

\begin{verbatim}
##  [1] "X"                   "studyName"           "Sample.Number"      
##  [4] "Species"             "Region"              "Island"             
##  [7] "Stage"               "Individual.ID"       "Clutch.Completion"  
## [10] "Date.Egg"            "Culmen.Length..mm."  "Culmen.Depth..mm."  
## [13] "Flipper.Length..mm." "Body.Mass..g."       "Sex"                
## [16] "Delta.15.N..o.oo."   "Delta.13.C..o.oo."   "Comments"
\end{verbatim}

\begin{Shaded}
\begin{Highlighting}[]
\CommentTok{\# Create a new variable called penguins\_clean, and remove two columns will be removed.The result is cleaned data.}
\NormalTok{penguins\_clean }\OtherTok{\textless{}{-}} \FunctionTok{select}\NormalTok{(penguins\_raw,}\SpecialCharTok{{-}}\FunctionTok{starts\_with}\NormalTok{(}\StringTok{"Delta"}\NormalTok{))}
\NormalTok{penguins\_clean }\OtherTok{\textless{}{-}} \FunctionTok{select}\NormalTok{(penguins\_clean,}\SpecialCharTok{{-}}\NormalTok{Comments)}

\CommentTok{\# Check the column names in the new data frame:}
\FunctionTok{names}\NormalTok{(penguins\_clean) }\CommentTok{\# the Delta is gone. }
\end{Highlighting}
\end{Shaded}

\begin{verbatim}
##  [1] "X"                   "studyName"           "Sample.Number"      
##  [4] "Species"             "Region"              "Island"             
##  [7] "Stage"               "Individual.ID"       "Clutch.Completion"  
## [10] "Date.Egg"            "Culmen.Length..mm."  "Culmen.Depth..mm."  
## [13] "Flipper.Length..mm." "Body.Mass..g."       "Sex"
\end{verbatim}

\begin{Shaded}
\begin{Highlighting}[]
\CommentTok{\# using  functions to clean and further process the data is advisable.}

\NormalTok{remove\_NA }\OtherTok{\textless{}{-}} \ControlFlowTok{function}\NormalTok{(penguins\_clean) \{}
\NormalTok{  penguins\_data }\SpecialCharTok{\%\textgreater{}\%}
    \FunctionTok{na.omit}\NormalTok{()}
\NormalTok{\}}

\CommentTok{\# Now it has been cleaned (in line with the lesson), adjust if needed to suit testing of the hypothesis.}

\FunctionTok{ggplot}\NormalTok{(}\AttributeTok{data=}\NormalTok{penguins\_clean, }\FunctionTok{aes}\NormalTok{(}\AttributeTok{x=}\NormalTok{Body.Mass..g., }\AttributeTok{y=}\NormalTok{Culmen.Depth..mm.)) }\SpecialCharTok{+}
  \FunctionTok{geom\_point}\NormalTok{(}\AttributeTok{colour =} \StringTok{"blue"}\NormalTok{)  }\CommentTok{\# produces an exploratory scatter plot for the explanatory and response variables with respect to the three penguin species.}
\end{Highlighting}
\end{Shaded}

\begin{verbatim}
## Warning: Removed 2 rows containing missing values (`geom_point()`).
\end{verbatim}

\includegraphics{ReproducibleScienceMT23HomeworkDec2023_files/figure-latex/Statistics pt1-1.pdf}

\begin{Shaded}
\begin{Highlighting}[]
\CommentTok{\#knitr::opts\_chunk$set(echo = TRUE)}
\end{Highlighting}
\end{Shaded}

\begin{Shaded}
\begin{Highlighting}[]
\CommentTok{\#Running a linear regression model, as learned earlier in the MBiol FHS course (3).}
\NormalTok{Penguin\_model1 }\OtherTok{\textless{}{-}} \FunctionTok{lm}\NormalTok{(Culmen.Depth..mm. }\SpecialCharTok{\textasciitilde{}}\NormalTok{ Body.Mass..g. , penguins\_clean, }\AttributeTok{na.rm=}\ConstantTok{TRUE}\NormalTok{)}
\end{Highlighting}
\end{Shaded}

\begin{verbatim}
## Warning: In lm.fit(x, y, offset = offset, singular.ok = singular.ok, ...) :
##  extra argument 'na.rm' will be disregarded
\end{verbatim}

\begin{Shaded}
\begin{Highlighting}[]
\NormalTok{Penguin\_model1}
\end{Highlighting}
\end{Shaded}

\begin{verbatim}
## 
## Call:
## lm(formula = Culmen.Depth..mm. ~ Body.Mass..g., data = penguins_clean, 
##     na.rm = TRUE)
## 
## Coefficients:
##   (Intercept)  Body.Mass..g.  
##     22.033946      -0.001162
\end{verbatim}

\begin{Shaded}
\begin{Highlighting}[]
\FunctionTok{summary}\NormalTok{(Penguin\_model1)}\CommentTok{\# allows you to view the output of the linear regression model. This output applies to the overall penguin population, not individual species.}
\end{Highlighting}
\end{Shaded}

\begin{verbatim}
## 
## Call:
## lm(formula = Culmen.Depth..mm. ~ Body.Mass..g., data = penguins_clean, 
##     na.rm = TRUE)
## 
## Residuals:
##     Min      1Q  Median      3Q     Max 
## -3.7437 -1.2235 -0.0116  1.2326  4.3468 
## 
## Coefficients:
##                 Estimate Std. Error t value Pr(>|t|)    
## (Intercept)   22.0339465  0.5036206   43.75   <2e-16 ***
## Body.Mass..g. -0.0011621  0.0001177   -9.87   <2e-16 ***
## ---
## Signif. codes:  0 '***' 0.001 '**' 0.01 '*' 0.05 '.' 0.1 ' ' 1
## 
## Residual standard error: 1.744 on 340 degrees of freedom
##   (2 observations deleted due to missingness)
## Multiple R-squared:  0.2227, Adjusted R-squared:  0.2204 
## F-statistic: 97.41 on 1 and 340 DF,  p-value: < 2.2e-16
\end{verbatim}

\begin{Shaded}
\begin{Highlighting}[]
\CommentTok{\# alternative way can be found here (4)}

\CommentTok{\#knitr::opts\_chunk$set(echo = TRUE)}
\end{Highlighting}
\end{Shaded}

\hypertarget{results-discussion}{%
\subsubsection{Results \& Discussion}\label{results-discussion}}

\begin{Shaded}
\begin{Highlighting}[]
\FunctionTok{ggplot}\NormalTok{(Penguin\_model1, }\FunctionTok{aes}\NormalTok{(}\AttributeTok{x =}\NormalTok{ Body.Mass..g. , }\AttributeTok{y =}\NormalTok{ Culmen.Depth..mm.)) }\SpecialCharTok{+}
  \FunctionTok{geom\_point}\NormalTok{() }\SpecialCharTok{+}
  \FunctionTok{geom\_smooth}\NormalTok{(}\AttributeTok{method =} \StringTok{"lm"}\NormalTok{) }\SpecialCharTok{+}
  \FunctionTok{labs}\NormalTok{(}\AttributeTok{x=}\StringTok{"Body mass of penguin (g)"}\NormalTok{, }\AttributeTok{y =} \StringTok{"Culmen Depth of penguin (mm)"}\NormalTok{) }\SpecialCharTok{+}
  \FunctionTok{ggtitle}\NormalTok{(}\StringTok{"Body mass and culmen depth for the Palmer Penguins"}\NormalTok{)  }\CommentTok{\#found appropriate way to add a title here (5)}
\end{Highlighting}
\end{Shaded}

\begin{verbatim}
## `geom_smooth()` using formula = 'y ~ x'
\end{verbatim}

\includegraphics{ReproducibleScienceMT23HomeworkDec2023_files/figure-latex/Plotting Results-1.pdf}

\begin{Shaded}
\begin{Highlighting}[]
\CommentTok{\# this plots the model on a labelled and titled plot. Having clear and concise but informative titles and labels is important to effectively communicate the data.}
\CommentTok{\# the grey area around the linear regression line represents the 95\% confidence interval (3). }

\CommentTok{\#knitr::opts\_chunk$set(echo = TRUE)}
\end{Highlighting}
\end{Shaded}

\hypertarget{results}{%
\section{\# \# \# Results}\label{results}}

The model run in the analysis provided an intercept of 22.04 (2 decimal
places) that had a standard error of 0.50 ( decimal places), a t-value
of 43.75 (un-rounded. 2 decimal places). The slope of the line was
estimated at -0.0012 (4 decimal places) with a standard error of 0.00012
(5 decimal places).

The F-statistic produced by the model is 97.41 on 1 and 340 degrees of
freedom.

The p-value was 2.2\^{}-16, which is less than 0.05.

As shown in the figure titled ``Body mass and culmen depth for the
Palmer Penguins'', the 95\% confidence interval visually lies close to
the line for much of the data, though it widens near to the ends of the
regression line.

A negative correlation is visually apparent from the same figure , and
is shown by the negative slope (m= -0.0012)of the line, between body
mass and culmen depth for the palmer penguin population.

\hypertarget{discussion}{%
\section{\# \# \# Discussion}\label{discussion}}

The results of a regression line with a negative slope suggest that
culmen depth does not increase with increasing body mass.This, and the
p-value of 2.2\^{}-16 being less than 0.05 enables rejection of the null
hypothesis that culmen depth does not change with increasing body mass.

The alternate hypothesis that depth of culmen does change with
increasing penguin body mass is accepted. However, given the change in
95\% confidence interval at the ends of the line and the spread of the
data in two clusters, further, more stringent testing is strongly
advocated. This would ideally have a species specific model for a focal
penguin species, or a model and analysis for each penguin species.

\hypertarget{conclusion}{%
\subsubsection{Conclusion}\label{conclusion}}

This simplified work has found that culmen depth in the palmer penguins
does change with the body mass of the penguins. For the sample
population in the palmer penguins, this was an decrease in culmen depth
with increase in body mass. However, skepticism is required with regard
to this conclusion, given the presence of three species and two large
clusters of data points.

Further work could examine body mass and culmen depth with respect to
individual species within a sample population or community of species,
including the palmer penguins group. Such work could feed into study on
selection pressures for given regions that have been or would be
sampled.

An alternative avenue is to investigate the matter within taxonomic
groups.

Even more substantial data sets and rigorous statistical methods would
prove indispensable in such a work from both avenues suggested.

\hypertarget{references}{%
\subsubsection{References}\label{references}}

\begin{enumerate}
\def\labelenumi{\arabic{enumi}.}
\tightlist
\item
  WWF (accessed 2023). Facts, {[}online{]}. Last accessed 06/12/2023.
  \url{https://www.worldwildlife.org/species/penguin}
\end{enumerate}

2.Borras,A., Pascual,J., \& Senar,J.C., (2000). What do different bill
measures measure and what is the best method to use in granivorous
birds? , Journal of Field Ornithology, 71(4), pp.~606-611.

\begin{enumerate}
\def\labelenumi{\arabic{enumi}.}
\setcounter{enumi}{2}
\item
  Bath,E., (2022). ``Linear regression'' {[}online{]}. University of
  Oxford, MBiol year 2 statistics:
  \url{https://canvas.ox.ac.uk/courses/172666/files/5209303?module_item_id=2029946}
\item
  Bevans,R., (2020). ``Linear regression in R A Step-by-Step Guide \&
  Examples'' {[}online{]}. Last accessed 06/12/2023.
  \url{https://www.scribbr.com/statistics/linear-regression-in-r/}
\end{enumerate}

5.Stack overflow (accessed 2023). ``add a title to a ggplot
{[}duplicate{]}'' . {[}online{]}. Last accessed
07/12/2023.https://stackoverflow.com/questions/64000030/add-a-tittle-to-a-ggplot

\hypertarget{question-3}{%
\section{\# Question 3}\label{question-3}}

\hypertarget{a}{%
\section{\# \# a)}\label{a}}

Github link (Candidate 1066450):

\hypertarget{b}{%
\section{\# \# b)}\label{b}}

Partner's Github link:

Note: questions 3.c and 3.d answered after exchange of Github links.

\hypertarget{c}{%
\section{\# \# c)}\label{c}}

\hypertarget{d}{%
\section{\# \# d)}\label{d}}

\end{document}
